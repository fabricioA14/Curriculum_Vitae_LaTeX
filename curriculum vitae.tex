\documentclass[a4paper,10pt]{article}
\usepackage[utf8]{inputenc}
\usepackage[english]{babel}
\usepackage{geometry}
\geometry{a4paper, margin=1in}
\usepackage{enumitem}
\usepackage{hyperref}

\begin{document}

\begin{center}
	\textbf{\huge Fabricio dos Anjos Santa Rosa}\\
	\vspace{2mm}
	Av. Gentil Bittencourt 3151, 66073-220, Belém, Pará, PA\\
	(91) 99126-6006\\
	\href{mailto:fabriciodosanjos14@gmail.com}{fabriciodosanjos14@gmail.com}\\
	29 years old, Brazilian, Single\\
	Available for travel and relocation
\end{center}

\noindent\rule{\linewidth}{0.8pt}

\section*{Position}
Data Scientist - R/Shiny developer

\section*{Summary of Qualifications}
Biologist graduated from the Federal University of Pará - UFPA (2017) with a master's degree in Aquatic Ecology and Fisheries - PPGEAP from UFPA (2019). I am currently in the final month of my PhD at PPGEAP, conducting studies with mitogenomics and metabarcoding associated with the ecology of Neotropical fish. I have experience in genomics, ecology, experimental design, and data analysis. I am interested in developing a career in bioinformatics.

\section*{Professional Experience}
\begin{itemize}[left=0pt]
	\item \textbf{Florestas Engenharia} (05/2018 to 11/2018)\\
	Company in the environmental consulting sector\\
	\textbf{\textit{Environmental Analyst of Ichthyofauna}} - I worked conducting field research and producing statistical models using a multivariate approach.
	
	\item \textbf{Sete Soluções e Tecnologia Ambiental} (08/2019 to 09/2019) \\
	Company in the environmental consulting sector\\
	\textbf{\textit{Environmental Analyst of Herpetofauna}} - I worked conducting field research and learned how to develop technical reports associated with ecological contexts with taxonomic, phylogenetic, and functional focus.
	
	\item \textbf{AITA Serviços de Engenharia} (10/2019 to 10/2019)\\
	Company in the environmental consulting sector\\
	\textbf{\textit{Environmental Analyst of Ichthyofauna}} - I worked conducting field research and learned how to organize each stage of data analysis in documents using the R markdown tool [\href{https://github.com/fabricioA14/Functional_Indexes}{1}].
	
	\item \textbf{University of Oslo (Norway)} (04/2023 to 07/2023)\\
	University renowned for methodologies associated with Next-Generation Sequencing (NGS)\\
	\textbf{\textit{Visiting Researcher}} - The main focus during my months as a visiting researcher was to learn hands-on how to conduct experiments with NGS from the perspective of the Illumina platform with High-seq and Mi-seq. During the stay, I developed a pipeline in R and Bash that enabled more accurate molecular identification of the data generated through various NGS approaches [\href{https://github.com/fabricioA14/BLAST}{2}]. I also gained additional experience with SLURM (Simple Linux Utility for Resource Management) for efficient resource management in a cluster environment.
\end{itemize}

\section*{Academic Background}
\begin{itemize}[left=0pt]
	\item \textbf{Bachelor's Degree in Biological Sciences (Teaching)}\\
	Federal University of Pará (UFPA) (04/2013 to 04/2017)\\
	\textbf{\textit{Skills acquired}} - During this period, I mainly developed soft skills such as time management and professional public speaking.
	
	\item \textbf{Master's Degree in Aquatic Ecology and Fisheries}\\
	UFPA (04/2017 to 04/2019)\\
	\textbf{\textit{Skills acquired}} - Fieldwork and organization/processing of data from various sources (classical barcoding associated with population genetics and metabarcoding).
	
	\item \textbf{PhD in Aquatic Ecology and Fisheries}\\
	UFPA (04/2020 to 04/2024)\\
	\textbf{\textit{Skills acquired}} - General bioinformatics (R, Bash, and Python). Practical experience with all stages of laboratory work and data analysis, including the generation of machine learning models through the lens of random forest algorithm [\href{https://www.linkedin.com/feed/update/urn:li:activity:7147371811264000000/}{3}], for land use prediction, and neural networks [\href{https://github.com/fabricioA14/Neural_Networks_Tensorflow}{4}] for classification of stream fish species based on functional traits. Practical experience with the development of an Open Source web application in R using Shiny [\href{https://github.com/fabricioA14/refDBdelimiter}{5}] (not yet officially published). Familiarity with mitochondrial genomics using the shotgun approach, covering everything from adapter removal to quality control, assembly, annotation, and data visualization. Significant experience with scientific writing in English (four published articles, two as first author), including agile and professional document formatting using LaTeX markup language [\href{https://github.com/fabricioA14/Curriculum_Vitae_LaTeX}{6}].
\end{itemize}

\section*{Languages}
\begin{itemize}[left=0pt]
	\item Conversation: Professional fluency
	\item Writing: Advanced
	\item Reading: Advanced
\end{itemize}

\section*{Additional Education}
\begin{itemize}[left=0pt]
	\item R for Conservation Biology (Instituto de Pesquisas Ecológicas - IPÊ)
	\item R for Data Analysis (Data Science Academy)
	\item Geoprocessing - Practical use of QGIS (Udemy)
	\item Complete Guide to TensorFlow for Deep Learning with Python (Udemy)
	\item NeuroEvolution of Augmenting Topologies NEAT Neural Networks (Udemy)
\end{itemize}

\end{document}
