\documentclass[a4paper,10pt]{article}
\usepackage[utf8]{inputenc}
\usepackage[portuguese]{babel}
\usepackage{geometry}
\geometry{a4paper, margin=1in}
\usepackage{enumitem}
\usepackage{hyperref}

\begin{document}
	
	\begin{center}
		\textbf{\huge Fabricio dos Anjos Santa Rosa}\\
		\vspace{2mm}
		Av. Gentil Bittencourt 3151, 66073-220, Belém, Pará, PA\\
		(91) 99126-6006\\
		\href{mailto:fabriciodosanjos14@gmail.com}{fabriciodosanjos14@gmail.com}\\
		28 anos, Brasileiro, Solteiro\\
		Disponibilidade para viagens e mudança
	\end{center}

	\noindent\rule{\linewidth}{0.8pt}
	
	\section*{Cargo}
	Bolsista TT em Bioinformática

	
	\section*{Resumo de Qualificações}
	Biólogo graduado pela Universidade Federal do Pará - UFPA (2017) com mestrado em Ecologia Aquática e Pesca - PPGEAP pela UFPA (2019). Atualmente estou nos últimos meses de doutorado no PPGEAP desenvolvendo estudos com metabarcoding associado à ecologia de peixes neotropicais. Possuo experiência em genética geral, ecologia, design experimental e análise de dados. Tenho interesse em desenvolver uma carreira na área de bioinformática e análise de dados.
	
	\section*{Experiência Profissional}
	\begin{itemize}[left=0pt]
		\item \textbf{Florestas Engenharia} (05/2018 a 11/2018)\\
		Empresa do segmento de consultoria ambiental\\
		\textbf{\textit{Analista ambiental de Ictiofauna}} - Trabalhei realizando pesquisa de campo e produção de modelos estatísticos utilizando a abordagem multivariada.
		
		\item \textbf{Sete Soluções e Tecnologia Ambiental} (08/2019 a 09/2019) \\
		Empresa do segmento de consultoria ambiental\\
		\textbf{\textit{Analista ambiental de Herpetofauna}} - Trabalhei realizando pesquisa de campo e aprendi como desenvolver relatórios técnicos associados a contextos ecológicos com enfoque taxonômico, filogenético e funcional.
		
		\item \textbf{AITA Serviços de Engenharia} (10/2019 a 10/2019)\\
		Empresa do segmento de consultoria ambiental\\
		\textbf{\textit{Analista ambiental de Ictiofauna}} - Trabalhei realizando pesquisa de campo e aprendi como organizar cada etapa da análise de dados em documentos, utilizando a ferramenta R markdown [\href{https://github.com/fabricioA14/Functional_Indexes}{1}].
		
		\item \textbf{Universitetet i Oslo (Norway)} (04/2023 a 07/2023)\\
		Universidade referência em metodologias associadas à Next-Generation Sequencing (NGS)\\
		\textbf{\textit{Pesquisador visitante}} - O foco principal nos meus meses como pesquisador visitante foi aprender na prática a realizar experimentos com NGS sob a ótica da plataforma Illumina com High-seq e Mi-seq. Durante a estadia desenvolvi um pipeline em R e Bash que possibilitou a identificação molecular mais acurada dos dados gerados através das diversas abordagens do NGS [\href{https://github.com/fabricioA14/BLAST}{2}]. Também adquiri experiência adicional com SLURM (Simple Linux Utility for Resource Management) para gerenciamento eficiente de recursos computacionais em ambiente de cluster.
	\end{itemize}
	
	\section*{Formação Acadêmica}
	\begin{itemize}[left=0pt]
		\item \textbf{Graduação em Ciências Biológicas (Licenciatura)}\\
		Universidade Federal do Pará (UFPA) (04/2013 a 04/2017)\\
		\textbf{\textit{Habilidades adquiridas}} - Durante esse período desenvolvi principalmente as soft skills de gestão de tempo e oratória profissional.
		
		\item \textbf{Mestrado em Ecologia Aquática e Pesca (CAPES 5)}\\
		UFPA (04/2017 a 04/2019)\\
		\textbf{\textit{Habilidades adquiridas}} - Trabalho de campo e organização/processamento de dados de diversas origens (barcoding clássico associado à genética de populações e metabarcoding).
		
		\item \textbf{Doutorado em Ecologia Aquática e Pesca (CAPES 5)}\\
		UFPA (04/2020 a 04/2024)\\
		\textbf{\textit{Habilidades adquiridas}} - Bioinformática geral (R, Bash e Phyton). Experiência prática com todas as etapas do trabalho de laboratório e com análise de dados, incluindo a geração de modelos de machine learning sob a ótica dos algoritmos de random forest [\href{https://www.linkedin.com/feed/update/urn:li:activity:7147371811264000000/}{3}], para predição de usos de terra e neural networks [\href{https://github.com/fabricioA14/Neural_Networks_Tensorflow}{4}] para classificação de espécies de peixes de riacho com base em características funcionais. Experiência prática com o desenvolvimento de funções e pacotes Open Source em linguagem R [\href{https://github.com/fabricioA14/refDBdelimiter}{5}] (ainda não publicado oficialmente). Familiaridade com genômica mitocondrial, utilizando a abordagem shotgun, abrangendo desde a remoção de adaptadores até o controle de qualidade, montagem, anotação e visualização de dados. Experiência significativa com escrita científica em inglês (quatro artigos publicados, sendo dois como primeiro autor), incluindo formatação ágil e profissional de documentos utilizando a linguagem de marcação LaTeX [\href{https://github.com/fabricioA14/Curriculum_Vitae_LaTeX}{6}].
	\end{itemize}
	
	\section*{Idiomas}
	\begin{itemize}[left=0pt]
		\item Conversação: Fluência profissional
		\item Escrita: Avançado
		\item Leitura: Avançado
	\end{itemize}
	
	\section*{Formação Complementar}
	\begin{itemize}[left=0pt]
		\item Curso de R para Biologia da Conservação (Instituto de Pesquisas Ecológicas - IPÊ)
		\item R for Data Analysis (Data Science Academy)
		\item Geoprocessamento - Uso prático do QGIS (Udemy)
		\item Complete Guide to TensorFlow for Deep Learning with Python (Udemy)
		\item NeuroEvolution of Augmenting Topologies NEAT Neural Networks (Udemy)
	\end{itemize}
	
\end{document}
